\section{Mappa e sue funzionalità}
La pagina della mappa è stata la sezione dell'applicazione più difficile da
implementare, perchè al suo interno risiedono numerosi servizi e funzioni che
hanno richiesto diverso tempo per essere sviluppati e inoltre è stato necessario
documentarsi per capire come utilizzare le API che la pagina utilizza.

\subsection{Scelta del servizio API}
In un primo momento si è pensato di utilizzare il servizio mappe di Google Maps
che, oltre ad essere uno dei più efficienti e meglio documentati, è anche
implementabile facilmente. Il problema è nato nel momento in cui Google ha
deciso di cambiare le proprie politiche di utilizzo delle API verso la fine del
2018. Prima di quel momento, sotto a un certo numero di richieste al servizio di
geolocalizzazione il programmatore poteva usare liberamente il codice e senza
necessità di registrazione, ma in seguito l'azienda di Mountain View ha deciso
che chiunque volesse utilizzare il proprio servizio mappe dovesse prima
registrare un proprio account ed inserire una carta di credito che eventualmente
pagasse mensilmente le risorse di cui si è fatto uso. Questo scenario ha portato
a prendere in cosiderazione altri gestori di mappe. La scelta è ricaduta su
MapBox, azienda emergente nel proprio campo. Il servizio era gratuito e permetteva
un agevole utlizzo senza registrazione ma il problema era formato
dall'implementazione vera e propria. Al contrario di Google, non esiste un
pacchetto software che implementi il codice MapBox e quindi era necessario fare
uso di richieste http tramite valori codificati all'interno di lunghi url.
Inoltre la fluidità della mappa non rispettava le direttive del committente
dott. Marco Aceti, spesso a seguito di un rapido movimento delle dita per
spostarsi in un'altra zona della mappa la schermata rimaneva per qualche secondo
completamente grigia, rendendo l'esperienza di utilizzo sicuramente peggiore.
Nel seguito, tra Gennaio e Febbraio 2019 è stata rilasciata la prima versione
ufficiale di Flutter (versione 1.0.0) e tra le tante novità spiccava la presenza
di un widget particolare, chiamato GoogleMap. Semplificando notevolmente
l'utlizzo delle mappe, tale widget presenta ottime prestazioni e facilità di
implementazione. Si è quindi deciso di dedicare del tempo nell'apprendere ogni
aspetto delle politiche di utlizzo delle API di Google, capendo quindi che,
facendo un numero di richieste minore di una soglia stabilita, non è necessario
pagare nulla, anche se si è registrata una carta di credito. Quindi la scelta è
ricaduta sulle API di Google e si è importato nel progetto il pacchetto
\verb|googlemap|.

\subsection{MapPage}

\subsection{Funzionalità}

