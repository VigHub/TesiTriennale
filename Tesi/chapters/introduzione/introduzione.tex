%\documentclass[../../Tesi.tex]{subfiles}

%\begin{document}

\section{Introduzione}
	Nel 2017 in Italia la produzione di veicoli a due ruote a pedalata assistita è aumentata
	del 48\% e il mercato dell'e-bike ha ottenuto un incremento del 
	19\%  per un totale di 148.000 unità vendute.
	\cite{biciElettrica}   
	\subsection{Obiettivi e origine del progetto}
	Scopo del progetto è creare un’applicazione che mostri all’utilizzatore, in
	base  alla propria posizione, le più vicine stazioni di ricarica per bici
	elettriche. Inoltre vengono forniti numerosi servizi agguntivi per
	facilitarne l'utilizzo e la personalizzazione.\\
	L'idea è nata da un incontro con il dott. Marco Aceti, biker per passione,
	nell'estate 2018. Durante il colloquio ha spiegato come il mercato delle
	bici elettriche sia in continua espansione, e che sempre più ciclisti
	decidono di passare dalla tradizionale bicicletta a quella con pedalata
	assistita. Dunque ha raccontato la sua idea: un'applicazione che mette a
	conoscenza dell'utente la posizione delle colonnine per ricaricare la
	propria bici.
	
	\subsection{Scelta del framework}
	Il committente dell'applicazione desiderava che il prodotto fosse 
	usufruibile  da quanti più utenti possibile, e dunque una delle richieste
	era che l'app fosse accessibile sia a dispositivi Apple, con il sistema
	operativo iOS, sia per dispositivi Android. \newline
	Per i sistemi Android non ci sarebbero stati problemi relativi al linguaggio
	da utilizzare in quanto si aveva già avuto esperienza di Java, dei suoi
	costrutti e della sua sintassi. Inoltre il programma Android Studio avrebbe
	ulteriormente semplificato le cose. Il problema nasceva nel momento in cui
	si era deciso di sviluppare un'app per iOS: la nostra conoscenza relativa
	ai
	linguaggi (Swift e C\verb|#|) era pressochè nulla e si sarebbe dovuto
	investire diverso tempo per impararne in modo appropriato l'utilizzo. 
	Inoltre per poter sviluppare un'applicazione che possa poi essere eseguita su un
	Iphone si deve essere in possesso di un calcolatore Apple (Macbook) in
	quanto è necessario il programma Xcode, funzionante solo su quest'ultimo; e
	questo strumento non faceva parte delle nostre risorse.
	Non
	sapendo come procedere, si è deciso di chiedere maggiori informazioni a
	diversi	professori del corso di Ingegneria Informatica e anche al gruppo
	Unibg Seclab. Proprio questi ultimi ci hanno indicato come la soluzione al
	problema fosse un (al tempo) nascente framework di Google in grado di creare
	app native per entrambe le piattaforme di interesse: Flutter. Il prossimo
	capitolo è dedicato alla comprensione e all'utilizzo di concetti base per
	capire com sviluppare software mediante questo framework.
	
	\subsection{Organizzazione e sviluppo}
	Lavorando in un team è stato necessario impiegare del tempo per scegliere i 
	tool relativi allo sviluppo in gruppo, in modo da organizzare il tutto nel 
	miglior modo possibile.

	\subsubsection{Git e Bitbucket}
	Bitbucket è uno strumento di gestione del codice Git. Un
	progetto sviluppato con questi due tool dunque
	non solo risiede fisicamente sulle macchine sulle quali si testa il
	software, ma anche sul cloud, in modo da avere maggiore affidabilità delle
	versioni (con una gestione efficace relativa alle modifiche) e potendo poi
	accedere con un qualunque calcolatore al progetto. Per aggiungere un nuovo
	file al progetto è necessario scrivere               
	\verb|git --add nomefile|, se invece occore applicare delle modifiche
	inserendo anche un commento per meglio comprendere il lavoro appena concluso
	basta digitare \verb|git commit -a| \verb|-m "messaggio"| (dove il parametro
	\verb|-a| sta a indicare che si vuole applicare a \textit{tutte} le
	modifiche e il parametro \verb|-m| che si desidera lasciare un commento).
	Volendo rendere dunque \textit{effettive} le modifiche è necessario dare
	il comando \verb|git push  origin| \verb|master| in modo tale da inserire nel branch
	\textit{master} la nuova porzione di codice. Per ottenere le modifiche
	aggiunte da un altro membro del team bisogna scrivere \verb|git pull|:
	dopo tale comando sul proprio calcolatore sono presenti tutti i file della
	repository aggiornati all'ultima versione.

	\subsubsection{Trello}
	Trello è un programma che permette in modo molto rapido e soprattutto in
	maniera intuitiva di organizzare le mansioni e i compiti di ogni membro del
	team. La pagina principale consiste in una grande bacheca sulla quale sono
	visibili delle \textit{schede}, ognuna con un titolo e relativa a un
	particolare gruppo. Per ogni scheda è poi possibile indicare il membro del
	team al quale è riferito il lavoro indicato, indicare una lista di azioni
	(in modo da vedere la percentuale di avanzamento di quella particolare
	scheda) e settare dei promemoria importanti che non devono essere persi.

	\subsubsection{Visual Studio Code}
	Il progetto conteneva numerosi file di estensione diversa. Infatti erano
	presenti file dart (per l'applicazione vera e propria), xml (per lavori
	specifici lato Android), plist (file descrittivo lato iOS), immagini e altri
	ancora. Si è quindi deciso di utilizzare un editor testuale il più generale
	possibile e non legato allo sviluppo di un particolare linguaggio. La
	scelta è ricaduta su Visual Studio Code, di proprietà Microsoft. Oltre a
	possedere "out of the box" funzionalità molto comode per lo sviluppo
	software, è estensibile con migliaia di pacchetti relativi a pressochè ogni
	linguaggio. Nello specifico l'estensione per Flutter non solo presenta
	autocompletamenti vari e molto dettagliati, ma anche una comoda funzione per
	fare il debugging del software gestendo il tutto con una barra che raccoglie i
	principali comandi. 

	\subsection{Struttura generale dell'applicazione}
	Di seguito viene riportato un accenno alla struttura, alle pagine
	dell'intera applicazione in modo da dare un contesto e dare una visione
	d'insieme al lettore. A partire dal terzo capitolo ogni pagina verrà poi
	analizzata nel dettaglio in base ai propri componenti e alle sue
	funzionalità, mostrando anche immagini e codice. \newline
	Al primo accesso all'utente viene mostrata una pagina di login nella quale
	si può inserire la propria mail e password oppure creare un nuovo account.
	Se l'autenticazione o la crezione di un nuovo utente sono andate a buon
	fine, viene mostrata la pagina principale, la mappa, che indica tutte le
	stazioni di ricarica, noleggio e manutenzione di bici elettriche presenti
	nel database. Oltre a possedere numerose funzioni che verranno descritte nel
	proprio capitolo, da questa pagina è possibile acedere all'inserimento di
	una nuova stazione (che deve essere in ultimo confermata del gestore del
	database), e si può infine arrivare alla pagina profilo, dove si possono
	settare impostazioni personali come la tipologia di mappa che si vuole
	visualizzare e cambiare la password, si possono vedere le stazioni aggiunte
	dall'utente attuale ed effettuare il logout, per tornare così alla pagina
	iniziale di login. 	
	
%\end{document}