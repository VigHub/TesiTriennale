\chapter{Conclusioni}
Il progetto ha mosso i primi passi nell'estate del 2018 ed è stato concluso nel
marzo 2019. Nel corso di questo intervallo di tempo sono stati affrontati
innumerevoli problemi, dovuti principalmente alla giovane età del framework e
alla limitata esperienza di chi doveva realizzare l'applicazione.
In particolare si fa riferimento alle molte ore passate a cambiare completamente la pagina della
mappa dopo che gli sviluppatori Google hanno rilasciato un aggiornamento del
pacchetto google\_maps che ha stravolto completamente il suo utilizzo. \`E però
doveroso dire che l'esperienza maturata in ambito sviluppo software è aumentata
notevolmente, in quanto si è visto da vicino quali possono essere le reali
problematiche nella realizzazione di un'applicazione. Tra queste si cita anche
la facilità con cui il committente può cambiare idea, rendendo vano il tempo
impiegato nella realizzazione di una particolare funzionalità che in seguito è
stata scartata. Si è imparato dunque a definire fin da subito gli obiettivi e le
modalità realizzative con il cliente e a sfruttare tutti i tool necessari per
migliorare esponenzialmente l'organizzazione per lavorare in un team. Ciò che
rende maggiormente soddisfatti nel vedere il progetto concluso è la
consapevolezza che in un futuro l'applicazione potrebbe essere venduta nei vari
store ufficiali Android e Apple, e tutto questo senza aver sviluppato due volte
codice identico prima in Java e poi in Swift. L'aver studiato a fondo il
framework Flutter concede la possibilità di poter sviluppare applicazioni in
ambito professionale ma anche in ambito privato. Infatti si ha la certezza che
se in futuro fosse necessario creare velocemente un'app per una funzione
specifica, si possederebbero tutti i requisiti per completare il lavoro senza
doversi rivolgere a terzi. Ciò che ha guidato gli autori del progetto è la
curiosità di imparare qualcosa di nuovo e di utile nel mondo dell'informatica, e
si spera che tale testo abbia stuzzicato, anche solo in parte, l'interesse del lettore.  